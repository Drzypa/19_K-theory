\documentclass[12pt]{article}
\usepackage{pmmeta}
\pmcanonicalname{Khomology}
\pmcreated{2013-03-22 12:57:46}
\pmmodified{2013-03-22 12:57:46}
\pmowner{mhale}{572}
\pmmodifier{mhale}{572}
\pmtitle{K-homology}
\pmrecord{6}{33330}
\pmprivacy{1}
\pmauthor{mhale}{572}
\pmtype{Topic}
\pmcomment{trigger rebuild}
\pmclassification{msc}{19K33}
\pmrelated{FredholmModule}
\pmrelated{KTheory}

\endmetadata

% this is the default PlanetMath preamble.  as your knowledge
% of TeX increases, you will probably want to edit this, but
% it should be fine as is for beginners.

% almost certainly you want these
\usepackage{amssymb}
\usepackage{amsmath}
\usepackage{amsfonts}

% used for TeXing text within eps files
%\usepackage{psfrag}
% need this for including graphics (\includegraphics)
%\usepackage{graphicx}
% for neatly defining theorems and propositions
%\usepackage{amsthm}
% making logically defined graphics
%%%\usepackage{xypic}

% there are many more packages, add them here as you need them

\newcommand*{\hilbert}[1][H]{\mathord{\mathcal{#1}}}

% define commands here
\begin{document}
K-homology is a homology theory on the category of compact Hausdorff spaces.
It classifies the elliptic pseudo-differential operators acting on the
vector bundles over a space.
In terms of $C^*$-algebras, it classifies the Fredholm modules over an algebra.

An operator homotopy between two Fredholm modules $(\hilbert,F_0,\Gamma)$ and $(\hilbert,F_1,\Gamma)$
is a norm continuous path of Fredholm modules, $t \mapsto (\hilbert,F_t,\Gamma)$, $t \in [0,1]$.
Two Fredholm modules are then equivalent if they are related by unitary transformations or operator homotopies.
The $K^0(A)$ group is the abelian group of equivalence classes
of even Fredholm modules over A.
The $K^1(A)$ group is the abelian group of equivalence classes
of odd Fredholm modules over A.
Addition is given by direct summation of Fredholm modules,
and the inverse of $(\hilbert, F, \Gamma)$ is $(\hilbert, -F, -\Gamma)$.

\begin{thebibliography}{10}
\bibitem{Inassaridze}
N.~Higson and J.~Roe, {\em Analytic K-homology}.
\newblock Oxford University Press, 2000.
\end{thebibliography}
%%%%%
%%%%%
\end{document}
