\documentclass[12pt]{article}
\usepackage{pmmeta}
\pmcanonicalname{FredholmModule}
\pmcreated{2013-03-22 12:57:43}
\pmmodified{2013-03-22 12:57:43}
\pmowner{mhale}{572}
\pmmodifier{mhale}{572}
\pmtitle{Fredholm module}
\pmrecord{6}{33329}
\pmprivacy{1}
\pmauthor{mhale}{572}
\pmtype{Definition}
\pmcomment{trigger rebuild}
\pmclassification{msc}{19K33}
\pmclassification{msc}{46L87}
\pmclassification{msc}{47A53}
\pmrelated{KHomology}

\usepackage{amssymb}
\usepackage{amsmath}
\usepackage{amsfonts}
\usepackage{amsthm}

% used for TeXing text within eps files
%\usepackage{psfrag}
% need this for including graphics (\includegraphics)
%\usepackage{graphicx}
% making logically defined graphics
%%%\usepackage{xypic}

% my maths package

\newcommand*{\Nset}{\mathbb{N}}
\newcommand*{\Zset}{\mathbb{Z}}
\newcommand*{\Qset}{\mathbb{Q}}
\newcommand*{\Rset}{\mathbb{R}}
\newcommand*{\Cset}{\mathbb{C}}
\newcommand*{\Hset}{\mathbb{H}}
\newcommand*{\Oset}{\mathbb{O}}
\newcommand*{\Bset}{\mathbb{B}}
\newcommand*{\Kset}{\mathbb{K}}
\newcommand*{\Sset}{\mathbb{S}}
\newcommand*{\Tset}{\mathbb{T}}
\newcommand*{\GLgrp}{\mathrm{GL}}
\newcommand*{\SLgrp}{\mathrm{SL}}
\newcommand*{\Ogrp}{\mathrm{O}}
\newcommand*{\SOgrp}{\mathrm{SO}}
\newcommand*{\Ugrp}{\mathrm{U}}
\newcommand*{\SUgrp}{\mathrm{SU}}
\newcommand*{\e}{\mathop{\mathrm{e}}\nolimits}
\newcommand*{\im}{\mathord{\mathrm{i}}}
\newcommand*{\identity}{\mathord{\mathrm{1\!\!\!\:I}}}
\newcommand*{\tr}{\mathop{\mathrm{tr}}}
\newcommand*{\Tr}{\mathop{\mathrm{Tr}}}
\renewcommand*{\d}{\mathrm{d}}
\newcommand*{\deriv}[2]{\frac{\d #1}{\d #2}}
\newcommand*{\pderiv}[2]{\frac{\partial #1}{\partial #2}}
\newcommand*{\fderiv}[2]{\frac{\delta #1}{\delta #2}}

% my noncommutative geometry package

\newcommand*{\algebra}[1][A]{\mathord{\mathcal{#1}}}
\newcommand*{\hilbert}[1][H]{\mathord{\mathcal{#1}}}
\newcommand*{\hilbmod}[1][E]{\mathord{\mathcal{#1}}}
\newcommand*{\Matrix}[2]{\mathord{\mathrm{M}_{#1}(#2)}}
\newcommand*{\dixmier}{\mathop{\mathrm{Tr}_\omega}}
\newcommand*{\Res}{\mathop{\mathrm{Res}}}
\newcommand*{\Wres}{\mathop{\mathrm{Wres}}}
\newcommand*{\Aut}{\mathop{\mathrm{Aut}}\nolimits}
\newcommand*{\Inn}{\mathop{\mathrm{Inn}}\nolimits}
\newcommand*{\Out}{\mathop{\mathrm{Out}}\nolimits}
\newcommand*{\Diff}{\mathop{\mathrm{Diff}}\nolimits}
\newcommand*{\Ker}{\mathop{\mathrm{Ker}}\nolimits}
\newcommand*{\Coker}{\mathop{\mathrm{Coker}}\nolimits}
\newcommand*{\Img}{\mathop{\mathrm{Im}}\nolimits}
\newcommand*{\End}{\mathop{\mathrm{End}}\nolimits}
\newcommand*{\spin}{\mathop{\mathrm{spin}}\nolimits}
\newcommand*{\Ind}{\mathop{\mathrm{Ind}}\nolimits}
\newcommand*{\KK}{\mathit{KK}}
\newcommand*{\HH}{\mathit{HH}}
\newcommand*{\HC}{\mathit{HC}}
\newcommand*{\ch}{\mathop{\mathrm{ch}}\nolimits}

% my environments

\newtheoremstyle{inlinedefn}{}{0pt}{}{}{\bfseries}{.}{0.5em}{}
\theoremstyle{inlinedefn}
\newtheorem{definition}{Definition}

\newtheoremstyle{break}{\baselineskip}{\baselineskip}{\itshape}{}{\bfseries}{}{\newline}{}
\theoremstyle{break}
\newtheorem{example}{Example}

% misc commands

\newcommand*{\defn}[1]{\textbf{#1}}
\begin{document}
Fredholm modules represent abstract elliptic pseudo-differential operators.

\begin{definition}
An \defn{odd Fredholm module} $(\hilbert,F)$ over a $C^*$-algebra $A$
is given by an involutive representation $\pi$ of $A$ on a Hilbert space $\hilbert$,
together with an operator $F$ on $\hilbert$ such that
$F = F^*$, $F^2 = \identity$ and $[F,\pi(a)] \in \Kset(\hilbert)$ for all $a \in A$.
\end{definition}

\begin{definition}
An \defn{even Fredholm module} $(\hilbert,F,\Gamma)$ is given by
an odd Fredholm module $(\hilbert,F)$
together with a $\Zset_2$-grading $\Gamma$ on $\hilbert$,
$\Gamma = \Gamma^*$, $\Gamma^2 = \identity$,
such that $\Gamma\pi(a) = \pi(a)\Gamma$ and $\Gamma F = -F\Gamma$.
\end{definition}

\begin{definition}
A Fredholm module is called \defn{degenerate} if $[F,\pi(a)] = 0$ for all $a \in A$.
Degenerate Fredholm modules are homotopic to the 0-module.
\end{definition}

% Examples

\begin{example}[Fredholm modules over $\Cset$]
An even Fredholm module $(\hilbert,F,\Gamma)$ over $\Cset$ is given by
\begin{eqnarray*}
\hilbert & = & \Cset^k\oplus\Cset^k
\quad\mbox{with\ }
\pi(a) = \left(\begin{array}{cc}
a\identity_k & 0 \\
0 & 0 \end{array}\right), \\
F & = & \left(\begin{array}{cc}
0 & \identity_k \\
\identity_k & 0 \end{array}\right), \\
\Gamma & = & \left(\begin{array}{cc}
\identity_k & 0 \\
0 & -\identity_k \end{array}\right).
\end{eqnarray*}
\end{example}
%%%%%
%%%%%
\end{document}
