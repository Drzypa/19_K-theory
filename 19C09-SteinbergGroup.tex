\documentclass[12pt]{article}
\usepackage{pmmeta}
\pmcanonicalname{SteinbergGroup}
\pmcreated{2013-03-22 16:44:38}
\pmmodified{2013-03-22 16:44:38}
\pmowner{dublisk}{96}
\pmmodifier{dublisk}{96}
\pmtitle{Steinberg group}
\pmrecord{4}{38968}
\pmprivacy{1}
\pmauthor{dublisk}{96}
\pmtype{Definition}
\pmcomment{trigger rebuild}
\pmclassification{msc}{19C09}

\endmetadata

% this is the default PlanetMath preamble.  as your knowledge
% of TeX increases, you will probably want to edit this, but
% it should be fine as is for beginners.

% almost certainly you want these
\usepackage{amssymb}
\usepackage{amsmath}
\usepackage{amsfonts}

% used for TeXing text within eps files
%\usepackage{psfrag}
% need this for including graphics (\includegraphics)
%\usepackage{graphicx}
% for neatly defining theorems and propositions
%\usepackage{amsthm}
% making logically defined graphics
%%%\usepackage{xypic} 

% there are many more packages, add them here as you need them

% define commands here

\begin{document}
Given an associative ring $R$ with identity, the Steinberg group $St(R)$ describes the minimal amount of relations between elementary matrices in $R$.

For $n \geq 3$, define $St_n(R)$ to be the free abelian group on symbols $x_{ij}(r)$ for $i,j$ distinct integers between $1$ and $n$, and $r \in R$, subject to the following relations:

$$x_{ij}(r)x_{ij}(s) = x_{ij}(r+s)$$

\begin{equation*}
[x_{ij},x_{kl}] =
\begin{cases} 1 & \text{if $j \neq k$ and $i \neq l$} \\
x_{il}(rs) & \text{if $j = k$ and $i \neq l$} \\
x_{kj}(-sr) & \text{if $j \neq k$ and $i = l.$}
\end{cases}
\end{equation*}

Note that if $e_{ij}(r)$ denotes the elementary matrix with one along the diagonal, and $r$ in the $(i,j)$ entry, then the $e_{ij}(r)$ also satisfy the above relations, giving a well defined morphism $St_n(R) \to E_n(R)$, where the latter is the group of elementary matrices.

Taking a colimit over $n$ gives the Steinberg group $St(R)$. The importance of the Steinberg group is that the kernel of the map $St(R) \to E(R)$ is the second algebraic $K$-group of the ring $R$, $K_2(R)$. This also coincides with the kernel of the Steinberg group. One can also show that the Steinberg group is the universal central extension of the group $E(R)$.
%%%%%
%%%%%
\end{document}
