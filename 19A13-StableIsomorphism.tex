\documentclass[12pt]{article}
\usepackage{pmmeta}
\pmcanonicalname{StableIsomorphism}
\pmcreated{2013-03-22 15:00:00}
\pmmodified{2013-03-22 15:00:00}
\pmowner{CWoo}{3771}
\pmmodifier{CWoo}{3771}
\pmtitle{stable isomorphism}
\pmrecord{4}{36707}
\pmprivacy{1}
\pmauthor{CWoo}{3771}
\pmtype{Definition}
\pmcomment{trigger rebuild}
\pmclassification{msc}{19A13}
\pmrelated{AlgebraicKTheory}
\pmdefines{stably isomorphic}
\pmdefines{stably free}

\endmetadata

% this is the default PlanetMath preamble.  as your knowledge
% of TeX increases, you will probably want to edit this, but
% it should be fine as is for beginners.

% almost certainly you want these
\usepackage{amssymb,amscd}
\usepackage{amsmath}
\usepackage{amsfonts}

% used for TeXing text within eps files
%\usepackage{psfrag}
% need this for including graphics (\includegraphics)
%\usepackage{graphicx}
% for neatly defining theorems and propositions
%\usepackage{amsthm}
% making logically defined graphics
%%%\usepackage{xypic}

% there are many more packages, add them here as you need them

% define commands here
\begin{document}
Let $R$ be a ring with unity 1.  Two left $R$-modules $M$ and $N$
are said to be \emph{stably isomorphic} if there exists a finitely
generated free $R$-module $R^n$ ($n\geq1$) such that $$M\oplus
R^n\cong N\oplus R^n.$$  A left $R$-module is said to be
\emph{stably free} if it is stably isomorphic to a finitely
generated free $R$-module.  In other words, $M$ is stably free if
$$M\oplus R^m\cong R^n$$ for some positive integers $m,n$.

\textbf{Remark} In the Grothendieck group $K_0(R)$ of a ring $R$
with 1, two finitely generated projective module representatives $M$
and $N$ such that $[M]=[N]\in K_0(R)$ iff they are stably isomorphic
to each other.  In particular, $[M]$ is the zero element in $K_0(R)$
iff it is stably free.
%%%%%
%%%%%
\end{document}
