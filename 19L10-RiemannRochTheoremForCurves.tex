\documentclass[12pt]{article}
\usepackage{pmmeta}
\pmcanonicalname{RiemannRochTheoremForCurves}
\pmcreated{2013-03-22 12:03:05}
\pmmodified{2013-03-22 12:03:05}
\pmowner{mathcam}{2727}
\pmmodifier{mathcam}{2727}
\pmtitle{Riemann-Roch theorem for curves}
\pmrecord{12}{31098}
\pmprivacy{1}
\pmauthor{mathcam}{2727}
\pmtype{Theorem}
\pmcomment{trigger rebuild}
\pmclassification{msc}{19L10}
\pmclassification{msc}{14H99}
\pmrelated{HurwitzGenusFormula}

\endmetadata

\usepackage{amssymb}
\usepackage{amsmath}
\usepackage{amsfonts}
\usepackage{graphicx}
%%%\usepackage{xypic}
\begin{document}
Let $C$ be a projective nonsingular curve over an algebraically closed field.  If $D$ is a divisor on $C$, then

$$\ell(D) - \ell(K-D) = {\rm deg}(D) + 1 - g$$

where $g$ is the genus of the curve, and $K$ is the canonical divisor ($\ell(K)=g$).  Here $\ell(D)$ denotes the dimension of the \PMlinkid{space of functions associated to a divisor}{SpaceOfFunctionsAssociatedToADivisor}.
%%%%%
%%%%%
%%%%%
\end{document}
