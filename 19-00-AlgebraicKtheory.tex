\documentclass[12pt]{article}
\usepackage{pmmeta}
\pmcanonicalname{AlgebraicKtheory}
\pmcreated{2013-03-22 13:31:32}
\pmmodified{2013-03-22 13:31:32}
\pmowner{mhale}{572}
\pmmodifier{mhale}{572}
\pmtitle{Algebraic K-theory}
\pmrecord{10}{34117}
\pmprivacy{1}
\pmauthor{mhale}{572}
\pmtype{Topic}
\pmcomment{trigger rebuild}
\pmclassification{msc}{19-00}
\pmclassification{msc}{18F25}
\pmrelated{KTheory}
\pmrelated{GrothendieckGroup}
\pmrelated{StableIsomorphism}

\usepackage{amssymb}
\usepackage{amsmath}
\usepackage{amsfonts}
\usepackage{amsthm}

% used for TeXing text within eps files
%\usepackage{psfrag}
% need this for including graphics (\includegraphics)
%\usepackage{graphicx}
% making logically defined graphics
%%%\usepackage{xypic}

% my maths package

\newcommand*{\Nset}{\mathbb{N}}
\newcommand*{\Zset}{\mathbb{Z}}
\newcommand*{\Qset}{\mathbb{Q}}
\newcommand*{\Rset}{\mathbb{R}}
\newcommand*{\Cset}{\mathbb{C}}
\newcommand*{\Hset}{\mathbb{H}}
\newcommand*{\Oset}{\mathbb{O}}
\newcommand*{\Bset}{\mathbb{B}}
\newcommand*{\Kset}{\mathbb{K}}
\newcommand*{\Sset}{\mathbb{S}}
\newcommand*{\Tset}{\mathbb{T}}
\newcommand*{\GLgrp}{\mathrm{GL}}
\newcommand*{\Egrp}{\mathrm{E}}
\newcommand*{\SLgrp}{\mathrm{SL}}
\newcommand*{\Ogrp}{\mathrm{O}}
\newcommand*{\SOgrp}{\mathrm{SO}}
\newcommand*{\Ugrp}{\mathrm{U}}
\newcommand*{\SUgrp}{\mathrm{SU}}
\newcommand*{\e}{\mathop{\mathrm{e}}\nolimits}
\newcommand*{\im}{\mathord{\mathrm{i}}}
\newcommand*{\identity}{\mathord{\mathrm{1\!\!\!\:I}}}
\newcommand*{\tr}{\mathop{\mathrm{tr}}}
\newcommand*{\Tr}{\mathop{\mathrm{Tr}}}
\renewcommand*{\d}{\mathrm{d}}
\newcommand*{\deriv}[2]{\frac{\d #1}{\d #2}}
\newcommand*{\pderiv}[2]{\frac{\partial #1}{\partial #2}}
\newcommand*{\fderiv}[2]{\frac{\delta #1}{\delta #2}}

% my noncommutative geometry package

\newcommand*{\algebra}[1][A]{\mathord{\mathcal{#1}}}
\newcommand*{\hilbert}[1][H]{\mathord{\mathcal{#1}}}
\newcommand*{\hilbmod}[1][E]{\mathord{\mathcal{#1}}}
\newcommand*{\Matrix}[2]{\mathord{\mathrm{M}_{#1}(#2)}}
\newcommand*{\dixmier}{\mathop{\mathrm{Tr}_\omega}}
\newcommand*{\Res}{\mathop{\mathrm{Res}}}
\newcommand*{\Wres}{\mathop{\mathrm{Wres}}}
\newcommand*{\Aut}{\mathop{\mathrm{Aut}}\nolimits}
\newcommand*{\Inn}{\mathop{\mathrm{Inn}}\nolimits}
\newcommand*{\Out}{\mathop{\mathrm{Out}}\nolimits}
\newcommand*{\Diff}{\mathop{\mathrm{Diff}}\nolimits}
\newcommand*{\Ker}{\mathop{\mathrm{Ker}}\nolimits}
\newcommand*{\Coker}{\mathop{\mathrm{Coker}}\nolimits}
\newcommand*{\Img}{\mathop{\mathrm{Im}}\nolimits}
\newcommand*{\End}{\mathop{\mathrm{End}}\nolimits}
\newcommand*{\spin}{\mathop{\mathrm{spin}}\nolimits}
\newcommand*{\Ind}{\mathop{\mathrm{Ind}}\nolimits}
\newcommand*{\KK}{\mathit{KK}}
\newcommand*{\HH}{\mathit{HH}}
\newcommand*{\HC}{\mathit{HC}}
\newcommand*{\ch}{\mathop{\mathrm{ch}}\nolimits}

% my category theory package

\newcommand*{\mathcat}[1]{\mathord{\mathbf{#1}}}
\newcommand*{\id}{\mathrm{id}}
\newcommand*{\op}{\mathrm{op}}
\newcommand*{\boxprod}{\mathbin{\square}}

% my environments

\newtheoremstyle{inlinedefn}{}{0pt}{}{}{\bfseries}{.}{0.5em}{}
\theoremstyle{inlinedefn}
\newtheorem{definition}{Definition}

\newtheoremstyle{break}{\baselineskip}{\baselineskip}{\itshape}{}{\bfseries}{}{\newline}{}
\theoremstyle{break}
\newtheorem{example}{Example}

% misc commands

\newcommand*{\defn}[1]{\textbf{#1}}
\begin{document}
Algebraic K-theory is a series of functors on the category of rings.
Broadly speaking, it classifies ring invariants, i.e.\ ring properties that are Morita invariant.

\textbf{The functor $K_0$}

Let $R$ be a ring and denote by $\Matrix{\infty}{R}$ the algebraic direct limit of matrix algebras $\Matrix{n}{R}$ under the embeddings
$\Matrix{n}{R} \to \Matrix{n+1}{R} : a \mapsto \left(\begin{array}{cc} a & 0 \\ 0 & 0 \end{array}\right)$.
The zeroth K-group of $R$, $K_0(R)$, is the Grothendieck group (abelian group of formal differences) of idempotents in $\Matrix{\infty}{R}$ up to similarity transformations.
Let $p \in \Matrix{m}{R}$ and $q \in \Matrix{n}{R}$ be two idempotents.
The sum of their equivalence classes $[p]$ and $[q]$ is the equivalence class of their direct sum:
$[p]+[q] = [p \oplus q]$ where $p \oplus q = \mathrm{diag}(p,q) \in \Matrix{m+n}{R}$.
Equivalently, one can work with finitely generated projective modules over $R$.

\textbf{The functor $K_1$}

Denote by $\GLgrp_\infty(R)$ the direct limit of general linear groups $\GLgrp_n(R)$ under the embeddings
$\GLgrp_n(R) \to \GLgrp_{n+1}(R) : g \mapsto \left(\begin{array}{cc} g & 0 \\ 0 & 1 \end{array}\right)$.
Give $\GLgrp_\infty(R)$ the direct limit topology, i.e.\
a subset $U$ of $\GLgrp_\infty(R)$ is open if and only if
$U \cap \GLgrp_n(R)$ is an open subset of $\GLgrp_n(R)$, for all $n$.
The first K-group of $R$, $K_1(R)$, is the abelianisation of $\GLgrp_\infty(R)$, i.e.\
\[
K_1(R) = \GLgrp_\infty(R)/[\GLgrp_\infty(R),\GLgrp_\infty(R)].
\]
Note that this is the same as $K_1(R) = H_1(\GLgrp_\infty(R), \Zset)$,
the first group homology group (with integer coefficients).

\textbf{The functor $K_2$}

Let $\Egrp_n(R)$ be the elementary subgroup of $\GLgrp_n(R)$.
That is, the group generated by the elementary $n\times n$ matrices $e_{ij}(r)$, $r\in R$,
where $e_{ij}(r)$ is the matrix with ones on the diagonals, the value $r$ in row $i$, column $j$
and zeros elsewhere.
Denote by $\Egrp_\infty(R)$ the direct limit of the $\Egrp_n(R)$ using the construction above (note $\Egrp_\infty(R)$ is a subgroup of $\GLgrp_\infty(R)$).
The second K-group of $R$, $K_2(R)$, is the second group homology group (with integer coefficients) of $\Egrp_\infty(R)$,
\[
K_2(R) = H_2(\Egrp_\infty(R), \Zset).
\]

\textbf{Higher K-functors}

Higher K-groups are defined using the Quillen plus construction,
\begin{equation}
K^{\mathrm{alg}}_n(R) = \pi_n(B\GLgrp_\infty(R)^+),
\end{equation}
where $B\GLgrp_\infty(R)$ is the classifying space of $\GLgrp_\infty(R)$.

Rough sketch of suspension:
\begin{equation}
\Sigma R = \Sigma\Zset \otimes_\Zset R
\end{equation}
where $\Sigma\Zset = C\Zset/J\Zset$.
The cone, $C\Zset$, is the set of infinite matrices with integral coefficients
that have a finite number of non-trivial elements on each row and column.
The ideal $J\Zset$ consists of those matrices that have only finitely many
non-trivial coefficients.
\begin{equation}
K_i(R) \cong K_{i+1}(\Sigma R)
\end{equation}


Algebraic K-theory has a product structure,
\begin{equation}
K_i(R) \otimes K_j(S) \to K_{i+j}(R \otimes S).
\end{equation}

\begin{thebibliography}{10}
\bibitem{Inassaridze}
H. Inassaridze, {\em Algebraic K-theory}.
\newblock Kluwer Academic Publishers, 1994.

\bibitem{Loday}
Jean-Louis Loday, {\em Cyclic Homology}.
\newblock Springer-Verlag, 1992.
\end{thebibliography}
%%%%%
%%%%%
\end{document}
