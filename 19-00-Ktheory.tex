\documentclass[12pt]{article}
\usepackage{pmmeta}
\pmcanonicalname{Ktheory}
\pmcreated{2013-03-22 12:58:06}
\pmmodified{2013-03-22 12:58:06}
\pmowner{mhale}{572}
\pmmodifier{mhale}{572}
\pmtitle{K-theory}
\pmrecord{17}{33338}
\pmprivacy{1}
\pmauthor{mhale}{572}
\pmtype{Topic}
\pmcomment{trigger rebuild}
\pmclassification{msc}{19-00}
\pmsynonym{Topological K-theory}{Ktheory}
\pmrelated{KHomology}
\pmrelated{AlgebraicKTheory}
\pmrelated{GrothendieckGroup}

\usepackage{amssymb}
\usepackage{amsmath}
\usepackage{amsfonts}
\usepackage{amsthm}

% used for TeXing text within eps files
%\usepackage{psfrag}
% need this for including graphics (\includegraphics)
%\usepackage{graphicx}
% making logically defined graphics
%%%\usepackage{xypic}

% my maths package

\newcommand*{\Nset}{\mathbb{N}}
\newcommand*{\Zset}{\mathbb{Z}}
\newcommand*{\Qset}{\mathbb{Q}}
\newcommand*{\Rset}{\mathbb{R}}
\newcommand*{\Cset}{\mathbb{C}}
\newcommand*{\Hset}{\mathbb{H}}
\newcommand*{\Oset}{\mathbb{O}}
\newcommand*{\Bset}{\mathbb{B}}
\newcommand*{\Kset}{\mathbb{K}}
\newcommand*{\Sset}{\mathbb{S}}
\newcommand*{\Tset}{\mathbb{T}}
\newcommand*{\GLgrp}{\mathrm{GL}}
\newcommand*{\SLgrp}{\mathrm{SL}}
\newcommand*{\Ogrp}{\mathrm{O}}
\newcommand*{\SOgrp}{\mathrm{SO}}
\newcommand*{\Ugrp}{\mathrm{U}}
\newcommand*{\SUgrp}{\mathrm{SU}}
\newcommand*{\e}{\mathop{\mathrm{e}}\nolimits}
\newcommand*{\im}{\mathord{\mathrm{i}}}
\newcommand*{\identity}{\mathord{\mathrm{1\!\!\!\:I}}}
\newcommand*{\tr}{\mathop{\mathrm{tr}}}
\newcommand*{\Tr}{\mathop{\mathrm{Tr}}}
\renewcommand*{\d}{\mathrm{d}}
\newcommand*{\deriv}[2]{\frac{\d #1}{\d #2}}
\newcommand*{\pderiv}[2]{\frac{\partial #1}{\partial #2}}
\newcommand*{\fderiv}[2]{\frac{\delta #1}{\delta #2}}

% my noncommutative geometry package

\newcommand*{\algebra}[1][A]{\mathord{\mathcal{#1}}}
\newcommand*{\hilbert}[1][H]{\mathord{\mathcal{#1}}}
\newcommand*{\hilbmod}[1][E]{\mathord{\mathcal{#1}}}
\newcommand*{\Matrix}[2]{\mathord{\mathrm{M}_{#1}(#2)}}
\newcommand*{\dixmier}{\mathop{\mathrm{Tr}_\omega}}
\newcommand*{\Res}{\mathop{\mathrm{Res}}}
\newcommand*{\Wres}{\mathop{\mathrm{Wres}}}
\newcommand*{\Aut}{\mathop{\mathrm{Aut}}\nolimits}
\newcommand*{\Inn}{\mathop{\mathrm{Inn}}\nolimits}
\newcommand*{\Out}{\mathop{\mathrm{Out}}\nolimits}
\newcommand*{\Diff}{\mathop{\mathrm{Diff}}\nolimits}
\newcommand*{\Ker}{\mathop{\mathrm{Ker}}\nolimits}
\newcommand*{\Coker}{\mathop{\mathrm{Coker}}\nolimits}
\newcommand*{\Img}{\mathop{\mathrm{Im}}\nolimits}
\newcommand*{\End}{\mathop{\mathrm{End}}\nolimits}
\newcommand*{\spin}{\mathop{\mathrm{spin}}\nolimits}
\newcommand*{\Ind}{\mathop{\mathrm{Ind}}\nolimits}
\newcommand*{\KK}{\mathit{KK}}
\newcommand*{\HH}{\mathit{HH}}
\newcommand*{\HC}{\mathit{HC}}
\newcommand*{\ch}{\mathop{\mathrm{ch}}\nolimits}

% my category theory package

\newcommand*{\mathcat}[1]{\mathord{\mathbf{#1}}}
\newcommand*{\id}{\mathrm{id}}
\newcommand*{\op}{\mathrm{op}}
\newcommand*{\boxprod}{\mathbin{\square}}

% my environments

\newtheoremstyle{inlinedefn}{}{0pt}{}{}{\bfseries}{.}{0.5em}{}
\theoremstyle{inlinedefn}
\newtheorem{definition}{Definition}

\newtheoremstyle{break}{\baselineskip}{\baselineskip}{\itshape}{}{\bfseries}{}{\newline}{}
\theoremstyle{break}
\newtheorem{example}{Example}

% misc commands

\newcommand*{\defn}[1]{\textbf{#1}}
\begin{document}
Topological K-theory is a generalised cohomology theory on the category
of compact Hausdorff spaces.
It classifies the vector bundles over a space $X$ up to stable equivalences.
Equivalently, via the Serre-Swan theorem, it classifies the finitely generated projective modules over the $C^*$-algebra $C(X)$.

Let $A$ be a unital $C^*$-algebra over $\Cset$ and denote by $\Matrix{\infty}{A}$ the algebraic direct limit of matrix algebras $\Matrix{n}{A}$ under the embeddings
$\Matrix{n}{A} \to \Matrix{n+1}{A} : a \mapsto \left(\begin{array}{cc} a & 0 \\ 0 & 0 \end{array}\right)$.
Identify the completion of $\Matrix{\infty}{A}$ with the stable algebra $A\otimes\Kset$ (where $\Kset$ is the compact operators on $l_2(\Nset)$),
which we will continue to denote by $\Matrix{\infty}{A}$.
The $K_0(A)$ group is the Grothendieck group (abelian group of formal differences) of the homotopy classes of the projections in $\Matrix{\infty}{A}$.
Two projections $p$ and $q$ are homotopic if there exists a norm continuous path of projections from $p$ to $q$.
Let $p \in \Matrix{m}{A}$ and $q \in \Matrix{n}{A}$ be two projections.
The sum of their homotopy classes $[p]$ and $[q]$ is the homotopy class of their direct sum:
$[p]+[q] = [p \oplus q]$ where $p \oplus q = \mathrm{diag}(p,q) \in \Matrix{m+n}{A}$.
Alternatively, one can consider equivalence classes of projections up to unitary transformations.
Unitary equivalence coincides with homotopy equivalence in $\Matrix{\infty}{A}$ (or $\Matrix{n}{A}$ for $n$ large enough).

Denote by $\Ugrp_\infty(A)$ the direct limit of unitary groups $\Ugrp_n(A)$ under the embeddings
$\Ugrp_n(A) \to \Ugrp_{n+1}(A) : u \mapsto \left(\begin{array}{cc} u & 0 \\ 0 & 1 \end{array}\right)$.
Give $\Ugrp_\infty(A)$ the direct limit topology, i.e.\
a subset $U$ of $\Ugrp_\infty(A)$ is open if and only if
$U \cap \Ugrp_n(A)$ is an open subset of $\Ugrp_n(A)$, for all $n$.
The $K_1(A)$ group is the Grothendieck group (abelian group of formal differences) of the homotopy classes of the unitaries in $\Ugrp_\infty(A)$.
Two unitaries $u$ and $v$ are homotopic if there exists a norm continuous path of unitaries from $u$ to $v$.
Let $u \in \Ugrp_m(A)$ and $v \in \Ugrp_n(A)$ be two unitaries.
The sum of their homotopy classes $[u]$ and $[v]$ is the homotopy class of their direct sum:
$[u]+[v] = [u \oplus v]$ where $u \oplus v = \mathrm{diag}(u,v) \in \Ugrp_{m+n}(A)$.
Equivalently, one can work with invertibles in $\GLgrp_\infty(A)$
(an invertible $g$ is connected to the unitary $u = g|g|^{-1}$ via the homotopy $t \to g|g|^{-t}$).

Higher K-groups can be defined through repeated suspensions,
\begin{equation}
K_n(A) = K_0(S^n A).
\end{equation}
But, the Bott periodicity theorem means that
\begin{equation}
K_1(SA) \cong K_0(A).
\end{equation}

The main properties of $K_i$ are:
\begin{eqnarray}
K_i(A \oplus B) & = & K_i(A) \oplus K_i(B), \\
K_i(\Matrix{n}{A}) & = & K_i(A) \quad\mbox{(Morita invariance)}, \\
K_i(A \otimes \Kset) & = & K_i(A) \quad\mbox{(stability)}, \\
K_{i+2}(A) & = & K_i(A) \quad\mbox{(Bott periodicity)}.
\end{eqnarray}

There are three flavours of topological K-theory to handle the cases
of $A$ being complex (over $\Cset$), real (over $\Rset$) or Real
(with a given real structure).
\begin{eqnarray}
K_i(C(X,\Cset)) & = & \mathit{KU}^{-i}(X) \quad\mbox{(complex/unitary)}, \\
K_i(C(X,\Rset)) & = & \mathit{KO}^{-i}(X) \quad\mbox{(real/orthogonal)}, \\
\mathit{KR}_i(C(X),J) & = & \mathit{KR}^{-i}(X,J) \quad\mbox{(Real)}.
\end{eqnarray}

Real K-theory has a Bott period of 8, rather than 2.

\begin{thebibliography}{10}
\bibitem{Wegge-Olsen}
N.~E. Wegge-Olsen, {\em K-theory and $C^*$-algebras}.
\newblock Oxford science publications. Oxford University Press, 1993.

\bibitem{Blackadar}
B.~Blackadar, {\em K-Theory for Operator Algebras}.
\newblock Cambridge University Press, 2nd~ed., 1998.

\bibitem{Larsen}
M.~R{\o}rdam, F.~Larsen and N.~J.~Laustsen, {\em An Introduction to K-Theory for $C^*$-Algebras}.
\newblock Cambridge University Press, 2000.
\end{thebibliography}
%%%%%
%%%%%
\end{document}
